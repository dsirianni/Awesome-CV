%-------------------------------------------------------------------------------
%	SECTION TITLE
%-------------------------------------------------------------------------------
\cvsection{Teaching Experience}\\

\cvsubsection{Instructor of Record}

%-------------------------------------------------------------------------------
%	CONTENT
%-------------------------------------------------------------------------------
\begin{cventries}

%---------------------------------------------------------

  \cvteachingentry
    {CHEM 6481/6491 R (Upper-Division Undergraduate/Graduate Level)} % School, Course # (Explain Level of Course)
    {Mathematical Methods for Chemical Physics} % Course Title
    {School of Chemistry \& Biochemistry $\vert$ Georgia Tech} % Institute
    {Fall '16--'18} % Semesters
    {% Description & Duties
    \begin{cvdescription}
    \item[Course Description] This course surveys mathematical concepts commonly
            encountered in chemical physics.  Topics include complex analysis, linear
            algebra \& functional analysis, statistics, ordinary \& partial differential
            equations, and integral transformations.
    \item[Duties] Design course curriculum; write and lead course lectures; prepare 
            assignments to augment classroom discussion; hold office hours to assist 
            students with challenging concepts.
    \end{cvdescription}
    }

%---------------------------------------------------------
\end{cventries}

\cvsubsection{Co-Instructor of Record}

%-------------------------------------------------------------------------------
%	CONTENT
%-------------------------------------------------------------------------------
\begin{cventries}

%---------------------------------------------------------

  \cvteachingentry
    {CHEM 4803/8843 DR (Upper-Division Undergraduate/Graduate Level)} % School, Course # (Explain Level of Course)
    {Special Topics: Python for Data Science} % Course Title
    {School of Chemistry \& Biochemistry $\vert$ Georgia Tech} % Institute
    {Fall '19} % Semesters
    {% Description & Duties
    \begin{cvdescription}
    \item[Course Description] Students learn the basic principles of Data Science
        and develop skills working with the most common tools in the world of Data
        Science, building from foundational experience with computer programming in the
        highly versatile Python language. The knowledge and skills developed in this
        course will therefore be transferable directly to students’ future careers in
        the science, technology, or business sectors.
    \item[Duties] Collaborate with co-instructor to design course curriculum, write and 
        present course lectures, and prepare projects and assignments to augment classroo
        discussion; hold office hours to assist students with challenging concepts.
    \end{cvdescription}
    }

%---------------------------------------------------------
\end{cventries}

\cvsubsection{Substitute/Guest Lecturer}

%-------------------------------------------------------------------------------
%	CONTENT
%-------------------------------------------------------------------------------
\begin{cventries}

%---------------------------------------------------------

  \cvteachingentry
    {CHEM 6491 (Graduate Level)} % School, Course # (Explain Level of Course)
    {Quantum Mechanics} % Course Title
    {School of Chemistry \& Biochemistry $\vert$ Georgia Tech} % Institute
    {Fall '16--'18} % Semesters
    {% Description & Duties
    \begin{cvdescription}
    \item[Course Description] Important concepts and applications of quantum mechanics
        at the intermediate level, including operators, perturbation and variational
        methods applied to atoms and molecules.
    \item[Duties] Lead several 50-minute lectures to $\sim$15 graduate students, covering
        topics including the time independent Schr\"odinger equation, the postulates of
        quantum mechanics, the Dirac delta function and momentum space, and extensions
        of approximate methods to many electron systems.
    \end{cvdescription}
    }

%---------------------------------------------------------
  \cvteachingentry
    {CHEM 6485 (Graduate Level)} % School, Course # (Explain Level of Course)
    {Computational Chemistry} % Course Title
    {School of Chemistry \& Biochemistry $\vert$ Georgia Tech} % Institute
    {Spring '15--'19} % Semesters
    {% Description & Duties
    \begin{cvdescription}
    \item[Course Description] Introductory course in computational chemistry, discussing
        electronic structure theory, semiemphirical methods, molecular mechanics, 
        transistion-state searching, and computation of thermodynamic quantities. 
    \item[Duties] Lead several 50-minute lectures to $\sim$25 graduate students, covering
        topics including the Born--Oppenheimer approximation and potential energy surfaces,
        the Hartree--Fock equations, basis sets, static and dynamical electron correlation,
        and molecular properties.
    \end{cvdescription}
    }

%---------------------------------------------------------
  \cvteachingentry
    {CHEM 3412 (Upper-Division Undergraduate Level)} % School, Course # (Explain Level of Course)
    {Physical Chemistry II} % Course Title
    {School of Chemistry \& Biochemistry $\vert$ Georgia Tech} % Institute
    {Spring '15} % Semesters
    {% Description & Duties
    \begin{cvdescription}
    \item[Course Description] Quantum mechanics, atomic and molecular structure, bonding
        theory, molecular spectroscopy, statistical mechanics. 
    \item[Duties] Lead two 50-minute lectures to $\sim$130 junior- and senior-level 
        undergraduate students, covering topics including the ladder-operator solution
        to the quantum harmonic oscillator, degeneracy, and $p$-orbital splitting via the
        Stark effect.
    \end{cvdescription}
    }

%---------------------------------------------------------
  \cvteachingentry
    {IDEaS Bootcamp (Mixed Undergraduate/Graduate Level)} % School, Course # (Explain Level of Course)
    {Summer Data Science Bootcamp} % Course Title
    {Institute for Data Engineering and Science $\vert$ Georgia Tech} % Institute
    {Summer '19} % Semesters
    {% Description & Duties
    \begin{cvdescription}
    \item[Course Description] This one-week bootcamp provides undergraduate and graduate
        students in science and engineering who have an introductory-level familiarity 
        with any computer programming language an introduction to data management and 
        visualization, data modeling, deep learning, and scientific programming in the
        Python programming language.
    \item[Duties] Lead one 30-minute lecture to a mixed audience of $\sim$80
        undergraduate and graduate students covering the application of data science and
        deep learning to open research questions in the fields of quantum chemistry and 
        electronic structure theory.
    \end{cvdescription}
    }

%---------------------------------------------------------
  \cvteachingentry
    {IDEaS Workshop (Mixed Undergraduate/Graduate/Postgraduate Level)} % School, Course # (Explain Level of Course)
    {Summer Workshop in Data Science \& Scientific Computing} % Course Title
    {Institute for Data Engineering and Science $\vert$ Georgia Tech} % Institute
    {Summer '18} % Semesters
    {% Description & Duties
    \begin{cvdescription}
    \item[Course Description] This five-week workshop engages undergraduates,
        graduate students, and postdocs/professionals in the comptational sciences,
        natural sciences, and engineering disciplines to provide an introduction to
        scientific computing and programming with emphasis on the topics of scientific
        computing using the Python programming language, numerical linear algebra,
        databases, and machine learning.
    \item[Duties] Lead one 50-minute lecture to a mixed audience of $\sim$60
        undergraduate, graduate, and professionalstudents covering (i) variable scope
        and namespaces in the Python programming language and (ii) the basic functionality 
        and usage of several advanced libraries for scientific computing in Python.
    \end{cvdescription}
    }

%---------------------------------------------------------
\end{cventries}

\cvsubsection{Graduate Teaching Assistant}

%-------------------------------------------------------------------------------
%	CONTENT
%-------------------------------------------------------------------------------
\begin{cventries}

%---------------------------------------------------------
  \cvteachingentry
    {IDEaS Bootcamp (Mixed Undergraduate/Graduate Level)} % School, Course # (Explain Level of Course)
    {Summer Data Science Bootcamp} % Course Title
    {Institute for Data Engineering and Science $\vert$ Georgia Tech} % Institute
    {Summer '19} % Semesters
    {% Description & Duties
    \begin{cvdescription}
    \item[Course Description] This one-week bootcamp provides undergraduate and graduate
        students in science and engineering who have an introductory-level familiarity 
        with any computer programming language an introduction to data management and 
        visualization, data modeling, deep learning, and scientific programming in the
        Python programming language.
    \item[Duties] Collaborate with instructors on developing
        interactive classroom activities and out-of-class assignemnts which target the
        students' development of desired knowledge and skills; lead classroom
        activities in interactive sessions; develop and implement virtual learning and
        collaboration environments for both students and instructors; provide targeted
        feedback to students on assignments via code review.
    \end{cvdescription}
    }

%---------------------------------------------------------
  \cvteachingentry
    {IDEaS Workshop (Mixed Undergraduate/Graduate/Postgraduate Level)} % School, Course # (Explain Level of Course)
    {Summer Workshop in Data Science \& Scientific Computing} % Course Title
    {Institute for Data Engineering and Science $\vert$ Georgia Tech} % Institute
    {Summer '18} % Semesters
    {% Description & Duties
    \begin{cvdescription}
    \item[Course Description] This five-week workshop engages undergraduates,
        graduate students, and postdocs/professionals in the comptational sciences,
        natural sciences, and engineering disciplines to provide an introduction to
        scientific computing and programming with emphasis on the topics of scientific
        computing using the Python programming language, numerical linear algebra,
        databases, and machine learning.
    \item[Duties] Present a lecture with demonstrative code examples and
        lead one 50-minute lecture on advanced Python topics; design collaborative 
        classroom activities to support lecture content and augment instruction; 
        collaboratively write and grade out-of-class assignments and projects; 
        administrate online educational platforms and materials; coordinate with 
        teaching assistants to ensure timely design of materials and feedback on 
        assignments.
    \end{cvdescription}
    }

%---------------------------------------------------------
  \cvteachingentry
    {CHEM 3412 (Upper-Division Undergraduate Level)} % School, Course # (Explain Level of Course)
    {Physical Chemistry II} % Course Title
    {School of Chemistry \& Biochemistry $\vert$ Georgia Tech} % Institute
    {Spring '16} % Semesters
    {% Description & Duties
    \begin{cvdescription}
    \item[Course Description] Quantum mechanics, atomic and molecular structure, bonding
        theory, molecular spectroscopy, statistical mechanics. 
    \item[Duties] Write and hold review sessions for each exam which highlight the
        important concepts and materials from the unit; hold office hours to assist
        students with specific course material; grade course homeworks, examinations, 
        and projects; substitute lecturer.
    \end{cvdescription}
    }

%---------------------------------------------------------
  \cvteachingentry
    {CHEM 1211K (First-Year Undergraduate Level)} % School, Course # (Explain Level of Course)
    {General Chemistry I} % Course Title
    {School of Chemistry \& Biochemistry $\vert$ Georgia Tech} % Institute
    {Fall '15} % Semesters
    {% Description & Duties
    \begin{cvdescription}
    \item[Course Description] Topics to be covered include atomic structure, bonding, 
        properties of matter, thermodynamics and physical equilibria. Laboratory 
        exercises supplement the lecture material.
    \item[Duties] Lead two sections (24 students/section) in laboratory experiments; 
        demonstrate and teach proper safety and laboratory technique; introduce and 
        teach course content during pre-laboratory discussions; lead two sections (24
        students/section) of recitation; hold individual tutoring hours to assist 
        students with specific course material; grade laboratory reports, assignments, 
        quizzes, and practical examinations.
    \end{cvdescription}
    }

%---------------------------------------------------------
\end{cventries}
